\section[Exersice 1. More Ghosts]{Exercise 1\\ {\large More Ghosts}}

    To solve this, I just created a new script \texttt{ChaserGhostSpawner}, adding a new behaviour to the \texttt{WaypointControl}. With a toggle, the Ghost can be set as a \textit{chaser} or \textit{patroller}.

    Also, the Ghost spawns at  \texttt{void Start()} by using the reference to the \texttt{Waypoint (3)} transform.
    \begin{listing}[htbp]
    \begin{minted}{csharp}
        [SerializeField] private bool isChaserGhost;
        [Header("Chase Settings")]
        [SerializeField] private float destinationReachedThreshold = 1f;
        [Range(0.1f, 5f)]
        [SerializeField] private float patrolReachedThreshold = 0.5f;

        void Start()
        {
            transform.position = waypoint3Transform.position;
        }

        void Update()
        {
            // Debug.Log($"Remaining distance: {agent.remainingDistance}");
            if(isChaserGhost)
                ChaserGhost();
            else
                Patroller();
        }

        void ChaserGhost()
        {
            // Skip if still moving towards destination
            if (agent.pathPending || agent.remainingDistance > destinationReachedThreshold) return;

            if (!playerTransform.GetComponent<PlayerMovement>().ReturnIsSafe())
            {
                observer.transform.gameObject.SetActive(true);
                agent.SetDestination(playerTransform.position);
            }
            else
            {
                observer.transform.gameObject.SetActive(false);
                agent.SetDestination(waypoint3Transform.position);
            }
        }
    \end{minted}
    \caption{ChaserGhost.cs}
    \label{lst:chaserghost}
    \end{listing}

\newpage
\section[Exersice 2. Be Safe]{Exercise 2\\ {\large Be Safe}}

    I added new methods on John to check for a new \texttt{SafeZone} script. The \texttt{SafeZone} script is blank, but it's the easiest and fastest way to check if John is inside a safe zone.

    To make the Ghosts go to the spawn position, you can see the method \texttt{ChaserGhost()} in Exercise \ref{lst:chaserghost} (\textit{Line 33 and 34, on the else statement.})
    \begin{minted}[label=SafeZoneManager.cs, fontsize=\footnotesize]{csharp}
        [SerializeField] private bool isInSafeZone;

        private void OnTriggerEnter(Collider other)
        {
            if(other.TryGetComponent<SafeZone>(out _))
            {
                SetSafeZone(true);
            }
        }

        private void OnTriggerExit(Collider other)
        {
            if(other.TryGetComponent<SafeZone>(out _))
            {
                SetSafeZone(false);
            }
        }

        private void SetSafeZone(bool value)
        {
            isInSafeZone = value;
        }

        public bool ReturnIsSafe()
        {
            return isInSafeZone;
        }
    \end{minted}

% section i cannot survive
\newpage
\section[Exercise 3. I Cannot Survive]{Exercise 3\\ {\large I Cannot Survive}}

    I created a new script that basically has a reference to everything that the \texttt{WaypointControl.cs} needs to work; then at \texttt{void Start()}, I start a new infinite coroutine that spawns a Ghost every second.

    And to make the Ghosts be able to or not capture John, in the Exercise \ref{lst:chaserghost} (\textit{Lines 28 and 33, on the else statement.}) I just disable the PoV object when John is inside a safezone.
    \begin{minted}[label=ChaserGhost.cs]{csharp}
        public GameObject chaserGhostPrefab;
        public Transform waypoint3Transform;
        public Transform playerTransform;
        public GameEnding gameEnding;

        void Start()
        {
            StartCoroutine(SpawnGhosts());
        }

        IEnumerator SpawnGhosts()
        {
            while (true)
            {
                GameObject spawnedGhost = Instantiate(chaserGhostPrefab, waypoint3Transform.position, Quaternion.identity);

                // Setup WaypointControl
                WaypointControl waypointControl = spawnedGhost.GetComponent<WaypointControl>();
                waypointControl.playerTransform = GameObject.FindGameObjectWithTag("Player").transform;
                waypointControl.waypoint3Transform = waypoint3Transform;

                // Setup Observer
                Observer observer = spawnedGhost.GetComponentInChildren<Observer>();
                // find with tag
                observer.player = GameObject.FindGameObjectWithTag("Player").transform;
                observer.gameEnding = gameEnding;

                yield return new WaitForSeconds(1f);
            }
        }
    \end{minted}
% section new mechanic
\newpage
\section[Exercise 4. New Mechanic]{Exercise 4\\ {\large New Mechanic}}

    For the new mechanic, I added a simple dash.
    \begin{minted}[label=PlayerMovement.cs]{csharp}
        private bool isDashing = false;
        private bool canDash = true;
        private float dashTimeLeft;
        private float cooldownTimeLeft;
        ...

        void FixedUpdate()
        {
            ...

            if (!canDash)
            {
                cooldownTimeLeft -= Time.fixedDeltaTime;
                if (cooldownTimeLeft <= 0)
                {
                    canDash = true;
                }
            }

            if (Input.GetKey(KeyCode.LeftShift) && canDash && (horizontal != 0 || vertical != 0))
            {
                isDashing = true;
                canDash = false;
                dashTimeLeft = dashDuration;
                cooldownTimeLeft = dashCooldown;
            }

            if (isDashing)
            {
                dashTimeLeft -= Time.fixedDeltaTime;
                if (dashTimeLeft <= 0)
                {
                    isDashing = false;
                }
            }

            ...
        }

    \end{minted}